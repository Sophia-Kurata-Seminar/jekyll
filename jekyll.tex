% Options for packages loaded elsewhere
\PassOptionsToPackage{unicode}{hyperref}
\PassOptionsToPackage{hyphens}{url}
%
\documentclass[
]{book}
\usepackage{lmodern}
\usepackage{amssymb,amsmath}
\usepackage{ifxetex,ifluatex}
\ifnum 0\ifxetex 1\fi\ifluatex 1\fi=0 % if pdftex
  \usepackage[T1]{fontenc}
  \usepackage[utf8]{inputenc}
  \usepackage{textcomp} % provide euro and other symbols
\else % if luatex or xetex
  \usepackage{unicode-math}
  \defaultfontfeatures{Scale=MatchLowercase}
  \defaultfontfeatures[\rmfamily]{Ligatures=TeX,Scale=1}
\fi
% Use upquote if available, for straight quotes in verbatim environments
\IfFileExists{upquote.sty}{\usepackage{upquote}}{}
\IfFileExists{microtype.sty}{% use microtype if available
  \usepackage[]{microtype}
  \UseMicrotypeSet[protrusion]{basicmath} % disable protrusion for tt fonts
}{}
\makeatletter
\@ifundefined{KOMAClassName}{% if non-KOMA class
  \IfFileExists{parskip.sty}{%
    \usepackage{parskip}
  }{% else
    \setlength{\parindent}{0pt}
    \setlength{\parskip}{6pt plus 2pt minus 1pt}}
}{% if KOMA class
  \KOMAoptions{parskip=half}}
\makeatother
\usepackage{xcolor}
\IfFileExists{xurl.sty}{\usepackage{xurl}}{} % add URL line breaks if available
\IfFileExists{bookmark.sty}{\usepackage{bookmark}}{\usepackage{hyperref}}
\hypersetup{
  pdftitle={Jekyllマニュアル},
  pdfauthor={Sophia-kurata-seminar},
  hidelinks,
  pdfcreator={LaTeX via pandoc}}
\urlstyle{same} % disable monospaced font for URLs
\usepackage{longtable,booktabs}
% Correct order of tables after \paragraph or \subparagraph
\usepackage{etoolbox}
\makeatletter
\patchcmd\longtable{\par}{\if@noskipsec\mbox{}\fi\par}{}{}
\makeatother
% Allow footnotes in longtable head/foot
\IfFileExists{footnotehyper.sty}{\usepackage{footnotehyper}}{\usepackage{footnote}}
\makesavenoteenv{longtable}
\usepackage{graphicx,grffile}
\makeatletter
\def\maxwidth{\ifdim\Gin@nat@width>\linewidth\linewidth\else\Gin@nat@width\fi}
\def\maxheight{\ifdim\Gin@nat@height>\textheight\textheight\else\Gin@nat@height\fi}
\makeatother
% Scale images if necessary, so that they will not overflow the page
% margins by default, and it is still possible to overwrite the defaults
% using explicit options in \includegraphics[width, height, ...]{}
\setkeys{Gin}{width=\maxwidth,height=\maxheight,keepaspectratio}
% Set default figure placement to htbp
\makeatletter
\def\fps@figure{htbp}
\makeatother
\setlength{\emergencystretch}{3em} % prevent overfull lines
\providecommand{\tightlist}{%
  \setlength{\itemsep}{0pt}\setlength{\parskip}{0pt}}
\setcounter{secnumdepth}{5}
\usepackage{booktabs}
\usepackage[]{natbib}
\bibliographystyle{apalike}

\title{Jekyllマニュアル}
\author{Sophia-kurata-seminar}
\date{2021-03-02}

\begin{document}
\maketitle

{
\setcounter{tocdepth}{1}
\tableofcontents
}
\hypertarget{ux4e8bux524dux6e96ux5099}{%
\chapter*{事前準備}\label{ux4e8bux524dux6e96ux5099}}
\addcontentsline{toc}{chapter}{事前準備}

\hypertarget{rubyjekyllux306eux30a4ux30f3ux30b9ux30c8ux30fcux30eb}{%
\section{Ruby,Jekyllのインストール}\label{rubyjekyllux306eux30a4ux30f3ux30b9ux30c8ux30fcux30eb}}

以下のサイトを参考に、インストールを進めてください。注意事項や重要事項だけ、以下にまとめたので、併せて確認してみて下さい。

参考:\url{http://jekyllrb-ja.github.io/docs/installation/windows/}

RubyInstallerをインストールする

RubyInstallerのサイト:\url{https://rubyinstaller.org/downloads/}

RubyInstallerのサイトから、Rubyインストーラーをインストールする際には、【WITH DEVKIT】の中で、太字になっているバージョンをダウンロードすることをお勧めします。

JekyllとBundlerをインストールする

コマンドプロンプトを開き、

\begin{verbatim}
gem install jekyll bundler
\end{verbatim}

を入力すると、JekyllとBundlerがインストールされます。

以下のコマンドで、インストールできているか確認してみましょう。

\begin{verbatim}
jekyll -v
\end{verbatim}

\hypertarget{ux57faux790eux77e5ux8b58}{%
\section{基礎知識}\label{ux57faux790eux77e5ux8b58}}

ここまでインストールの仕方を説明してきましたが、ここからは用語の説明です。\\
参考:\url{https://qiita.com/oshou/items/6283c2315dc7dd244aef}

Ruby,jekyllとは

Rubyはプログラミング言語のひとつで、jekyllはこの言語で書かれています。jekyllとは、Rubyで書かれた(静的)サイトを作るソフトウェアのことで、テンプレートを使えば簡単にサイトを作成できます。

jekyllテンプレートのサイト:\url{https://jekyllthemes.io/github-pages-themes}

Gemとは

プログラミング言語は機能をまとめたライブラリーを作成していることが多くあり、RubyではそのライブラリーのことをGemと呼びます(Gemとは、Rでいうパッケージのことです。)使いたいgemをgemfileに記入すればそのgemの機能を使うことができます。

bundlerとは

Gem同士の依存関係を整理するものです。bundler自体もgemのひとつですが、以下のような場合に備えて使用します。

「GemAを使うために、GemBを使う。さらにGemBを使うためにはGemCを使う」という時、GemCがアップデートされるとAもBも使えなくなる場合があります。こういう場合に、bundlerはgem同士の互換性を保ちながら各gemの導入、管理を行ってくれます。

特に、共同作業を行う場合、自分と相手のGemのバージョンが異なると、エラーメッセージが表示されます。なので、定期的に以下のコマンドで、Gemの管理を行いましょう。
*またエラーメッセージが出たときも同様に、以下のコマンドを試してみてください。

\begin{verbatim}
bundle install //gemfileに書かれたgemを読み込む
bundle update  //gemのバージョン更新
\end{verbatim}

\hypertarget{jekyllux30670ux304bux3089ux30b5ux30a4ux30c8ux3092ux4f5cux308b}{%
\chapter{Jekyllで0からサイトを作る}\label{jekyllux30670ux304bux3089ux30b5ux30a4ux30c8ux3092ux4f5cux308b}}

Default themeを利用して、ゼロからサイトを作ってみましょう。

\hypertarget{githubux306bux30ecux30ddux30b8ux30c8ux30eaux3092ux4f5cux6210}{%
\section{GitHubにレポジトリを作成}\label{githubux306bux30ecux30ddux30b8ux30c8ux30eaux3092ux4f5cux6210}}

\hypertarget{jekyllux30d5ux30a9ux30ebux30c0ux3092ux4f5cux6210ux3059ux308b}{%
\section{Jekyllフォルダを作成する}\label{jekyllux30d5ux30a9ux30ebux30c0ux3092ux4f5cux6210ux3059ux308b}}

コマンドプロンプトに以下のコードを書き込み、jekyllフォルダを作成してみましょう。

\begin{verbatim}
cd ~desktop
jekyll new フォルダ名 #Jekyllの構成ファイルが作成される
cd フォルダ名     #作成したJekyllフォルダに移動
\end{verbatim}

\hypertarget{config.ymlux306eux5909ux66f4}{%
\section{\_config.ymlの変更}\label{config.ymlux306eux5909ux66f4}}

作成したフォルダの中にある、\_config.ymlの変更を行います。(AtomやVS codeで開けます。)

\begin{verbatim}
baseurl:"/レポジトリ名"
url:"https://ユーザー名.github.io"
\end{verbatim}

\hypertarget{ux30b5ux30a4ux30c8ux306eux30d7ux30ecux30d3ux30e5ux30fcux3092ux8868ux793aux51faux529b}{%
\section{サイトのプレビューを表示&出力}\label{ux30b5ux30a4ux30c8ux306eux30d7ux30ecux30d3ux30e5ux30fcux3092ux8868ux793aux51faux529b}}

\begin{verbatim}
jekyll server --watch #プレビュー兼Build(_siteフォルダ内にhtmlが出力される)
 
jekyll build --d docs #docsフォルダ内に、htmlなどが出力*
\end{verbatim}

*デフォルトでは、\_siteフォルダ内に出力されますが、Github pagesではdocsフォルダが必要になるため、docsに出力します。

\hypertarget{githubux3078ux306epush}{%
\section{Githubへのpush}\label{githubux3078ux306epush}}

公開に向けて、Githubへのpushを行います。
参考:\url{https://sh-2.github.io/Rstudio_bookdown}

\begin{verbatim}
git init
git remote add orign URL
git add .
git commit- m"コメント"
git push origin master
\end{verbatim}

Githubから、Settingの変更を行う

\hypertarget{ux30c6ux30f3ux30d7ux30ecux30fcux30c8ux30c6ux30fcux30deux3092ux4f7fux3063ux3066webux30b5ux30a4ux30c8ux5236ux4f5c}{%
\chapter{テンプレートテーマを使ってWEBサイト制作}\label{ux30c6ux30f3ux30d7ux30ecux30fcux30c8ux30c6ux30fcux30deux3092ux4f7fux3063ux3066webux30b5ux30a4ux30c8ux5236ux4f5c}}

倉田ゼミのHPでは''just-the-doc''というテンプレートテーマを使用しています。
参考:\url{https://jekyll-themes.com/just-the-docs/}

\hypertarget{githubux3067ux30ecux30ddux30b8ux30c8ux30eaux3092ux4f5cux6210}{%
\section{Githubでレポジトリを作成}\label{githubux3067ux30ecux30ddux30b8ux30c8ux30eaux3092ux4f5cux6210}}

\hypertarget{jekyllux30d5ux30a9ux30ebux30c0ux3092ux4f5cux6210ux3059ux308b}{%
\section{Jekyllフォルダを作成する}\label{jekyllux30d5ux30a9ux30ebux30c0ux3092ux4f5cux6210ux3059ux308b}}

コマンドプロンプトに以下のコードを書き込み、jekyllフォルダを作成してみましょう。

\begin{verbatim}
cd ~desktop
jekyll new フォルダ名 #Jekyllの構成ファイルが作成される
\end{verbatim}

デフォルトのテーマを、選んだテーマ(just-the-docs)に変更しましょう。

\_postフォルダを削除し、\_config.yml,gemfile,マークダウンファイルを編集していきます。

\_config.yml

\begin{verbatim}
baseurl: "/リポジトリ名" 
url: "https://ユーザ名.github.io" 

theme: "just-the-docs"	#に上書き
\end{verbatim}

gemfile

\begin{verbatim}
gem "minima", "~> 2.5"	#消す
# gem "github-pages", group: :jekyll_plugins	#ここはコメントアウトしたままでいいかも。
gem "just-the-docs"	#config.ymlにremote themeを書く必要はないかも。themeのどちらかで良い?
\end{verbatim}

マークダウンファイル

VS codeで編集することができます。

VS codeのインストール

Visual Studio Code:\url{https://azure.microsoft.com/ja-jp/products/visual-studio-code/}

\#\# プレビュー&出力

\begin{verbatim}
cd ~フォルダ 		#jekyllのフォルダまで移動
bundle install	#gemfileに書かれたものをインストール
bundle update 	#gemfileの依存関係をアップデート?(定期的にした方がよい)
( bundle exec jekyll s)	#ローカル環境でプレビューできる。 & _siteフォルダにhtmlが出力される。
bundle exec jekyll buidl -d docs	#docsに結果を出力。実行する前にdocsを消す。でないとdocsが多層化される。
\end{verbatim}

\hypertarget{githubux306bpush}{%
\section{Githubにpush}\label{githubux306bpush}}

\begin{verbatim}
git init
git add .
git commit -m "コメント"
git remote add origin リポジトリURL
git push origin master
\end{verbatim}

github pagesのsettingをdocsに

\hypertarget{ux8af8ux6ce8ux610f}{%
\section{諸注意}\label{ux8af8ux6ce8ux610f}}

gemfile(やconfig,yml)をいじると色々なエラーが起こるときがある。
jekyll s(erver)は更新されないときがあるので、ファイルやターミナルを閉じてから再度やる。or無視してGithubにあげてしまうのもあり。

\hypertarget{ux5171ux540cux4f5cux696dux306bux3064ux3044ux3066}{%
\chapter{共同作業について}\label{ux5171ux540cux4f5cux696dux306bux3064ux3044ux3066}}

WEBサイトを共同で管理・制作していく方法を紹介します。

詳細な説明については、以下のURLのサイトを参考にしてください。

参考:\url{https://sh-2.github.io/Rstudio_bookdown}

\hypertarget{ux57faux672cux306eux64cdux4f5c}{%
\section{基本の操作}\label{ux57faux672cux306eux64cdux4f5c}}

管理者とAさんがGithubのリポジトリを利用して、共同でサイトの更新を行う場合の重要なポイントについて説明します。

管理者はこれまでと同様の手順でサイトを作成し、Githubにプッシュします。
Aさんの動きを見てみましょう。

まず、管理者が作成したリモートリポジトリを、自分のPC上にコピーします。

\begin{verbatim}
cd デスクトップ
git clone リポジトリのURL
\end{verbatim}

次にAさんは、自分が作業するブランチを作成します。

\begin{verbatim}
cd コピーしたフォルダ  #コピーしたリモートリポジトリのフォルダへ移動
git branch    #今あるブランチと、自分のいるブランチ(*)を表示
git branch branchA    #masterからbranchAを分岐させる。(masterにいる状態で実行しましょう)
git checkout branchA    #branchAに移動。今後はここで編集する。
git branch    #今いるブランチ(*)を確認(branchAにいるのか確認。)
\end{verbatim}

Aさんの作業環境が整いました。Aさんはマークダウンファイルの編集を行い、以下のコマンドをこれまで同様に実行します。

\begin{verbatim}
bundle exec jekyll server
bundle exec jekyll build -d docs
git add .
git commit -m"コメント"
\end{verbatim}

branchAの編集内容をリモートリポジトリにpushします。

\begin{verbatim}
git push origin branchA   #branchAの情報リモートリポジトリにアップロード
\end{verbatim}

Githubのリポジトリを確認すると、branchAというブランチを確認することができます。

続いて、プルリクエストとマージを行います。
branchAがpushされた後、GitHubを開くとpull requestを行うボタンが追加されています。押してみましょう。
編集内容についてコメントを残し、【Create pull request】を押します。
管理者はAさんのプルリクエストを確認することができます。

管理者は、masterとbranchAのマージを行う必要があります。プルリクエストされた画面にある【merge】】ボタンを押せば完了です。

*まれにコンフリクトを起こしている場合があります。Githubの案内に従って、コンフリクト内容を確認し、必要に応じて内容の削除を行ってください。

マージされたことを確認できたら、Aさんは最新のリモートリポジトリの情報をローカルに持ってきて、不要になったブランチを削除します。

\begin{verbatim}
git pull origin master   #ローカルをリモートの最新情報に更新
\end{verbatim}

\begin{verbatim}
git checkout master   #masterブランチに移動
git branch    #branchAの存在、masterにいることを確認
git branch -d branchA   #branchAの削除
git branch    #branchAが消えているのか確認
\end{verbatim}

2回目以降、Aさんが更新作業を行う場合や新たなメンバーが更新作業を共同で行う場合も、この方法で進めることができます。

\hypertarget{ux5171ux540cux4f5cux696dux30b3ux30f3ux30d5ux30eaux30afux30c8ux7de8}{%
\chapter{共同作業(コンフリクト編)}\label{ux5171ux540cux4f5cux696dux30b3ux30f3ux30d5ux30eaux30afux30c8ux7de8}}

共同作業を行っていると、コンフリクトが発生し、マージができない事態が発生することがあります。
本章では、コンフリクトが発生し、マージがうまくいかないときの対処法をご紹介します。

\hypertarget{ux5358ux7d14ux306aux30b3ux30f3ux30d5ux30eaux30afux30c8ux306eux5834ux5408}{%
\section{単純なコンフリクトの場合}\label{ux5358ux7d14ux306aux30b3ux30f3ux30d5ux30eaux30afux30c8ux306eux5834ux5408}}

Githubの画面上の【Resolve Conflicts】を押します。すると、以下のような画面が表示されるので、不要な部分を削除することでコンフリクトを解決しましょう。
(画像作成中)

着色された部分について、保存したい変更を残し、それ以外は削除します。

\hypertarget{ux8907ux96d1ux306aux30b3ux30f3ux30d5ux30eaux30afux30c8ux306eux5834ux5408}{%
\section{複雑なコンフリクトの場合}\label{ux8907ux96d1ux306aux30b3ux30f3ux30d5ux30eaux30afux30c8ux306eux5834ux5408}}

ブランチが3つ以上ある場合や、プルリクエストの前に複雑な作業を行った場合、先ほどのようにGithub(ブラウザ版)上で対処できないことがあります。
そんな時は以下の2つの方法を順に試してみてください。

\hypertarget{fetchux3068merge}{%
\subsection{FetchとMerge}\label{fetchux3068merge}}

\begin{verbatim}
git fetch
\end{verbatim}

上のコマンドは、リモートリポジトリから最新情報をローカルリポジトリに持ってくるものです。
pullとは異なり、ローカルリポジトリが更新されるだけでファイルは更新されません。

\begin{verbatim}
git merge
\end{verbatim}

fetchのあと、mergeすることで、ローカルのファイルが最新状態に更新されます。

\hypertarget{githubux30c7ux30b9ux30afux30c8ux30c3ux30d7ux7248ux306eux5229ux7528}{%
\subsection{Github(デスクトップ版)の利用}\label{githubux30c7ux30b9ux30afux30c8ux30c3ux30d7ux7248ux306eux5229ux7528}}

Fetch,Mergeをしてもうまくいかない場合は、Github(デスクトップ版)を利用しましょう。デスクトップ版では、複雑なコンフリクトであっても、ブラウザ版同様に、不要な部分を削除することで解決できます。

Github(デスクトップ版):\url{https://desktop.github.com/}

\hypertarget{final-words}{%
\chapter{Final Words}\label{final-words}}

We have finished a nice book.

  \bibliography{book.bib,packages.bib}

\end{document}
